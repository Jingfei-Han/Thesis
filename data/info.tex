% !Mode:: "TeX:UTF-8"

% 学院中英文名,中文不需要“学院”二字
% 院系英文名可从以下导航页面进入各个学院的主页查看
% http://www.buaa.edu.cn/xyykc/index.htm
\school
{计算机}{School of Computer Science \& Engineering}

% 专业中英文名
\major
{计算机应用技术}{Computer Science and Technology}

% 论文中英文标题
\thesistitle
{基于知识库和强化学习的代表性话题抽取研究}
{}
{Representative Topics Extraction Based on Knowledge Base and Reinforcement Learning}
{}

% 作者中英文名
\thesisauthor
{韩京飞}{Jingfei Han}

% 导师中英文名
\teacher
{荣文戈}{Rong Wenge}
% 副导师中英文名
% 注:慎用‘副导师’,见北航研究生毕业论文规范
%\subteacher{副导师}{subteacher}

% 中图分类号,可在 http://www.ztflh.com/ 查询
\category{TP391}

% 本科生为毕设开始时间;研究生为学习开始时间
\thesisbegin{2016}{09}{01}

% 本科生为毕设结束时间;研究生为学习结束时间
\thesisend{2019}{}{}

% 毕设答辩时间
\defense{2019}{}{}

% 中文摘要关键字
\ckeyword{层次概率,神经语言模型,递归神经网络,自然语言处理}

% 英文摘要关键字
\ekeyword{Hierarchical Softmax, Neural Language Model, Recurrent Neural Network, Natural Language Processing.}
% !Mode:: "TeX:UTF-8"

% 研究方向
\direction{自然语言处理}

% 导师职称中英文
\teacherdegree{副教授}{Associate Prof.}
% 副导师职称中英文
% 注:慎用‘副导师’,见北航研究生毕业论文规范
%\subteacherdegree{讲师}{Teacher}

% 保密等级,注:非保密论文时不需要此项
%保密论文请更改‘buaathesis.cls’里相应代码
%\mibao{机密}

%申请学位级别
\applydegree{工学硕士}

% 论文编号,由10006+学号组成
\thesisID{10006SY1606405}

% 论文提交时间
\commit{2019}{}

% 学位授予日期
\award{2019}{}{}
